\documentclass{article}
\usepackage[utf8]{inputenc}
\usepackage{multirow}
\usepackage{graphicx}
\usepackage{subfigure}
\usepackage{biblatex}
\usepackage{stix}
\usepackage{fdsymbol}

\title{Lab1}
\author{1905050 }
\date{November 2022}
\addbibresource{bib.bib}


\begin{document}


\newpage

\section{Introduction}
Bangladesh (/ˌbæŋɡləˈdɛʃ, ˌbɑːŋ-/;[12] Bengali: বাংলাদেশ, pronounced [ˈbaŋlaˌdeʃ] (listen)), officially the People's Republic of Bangladesh, is a country in South Asia. It is the eighth-most populous country in the world, with a population exceeding 165 million people in an area of 148,460 square kilometres (57,320 sq mi).[7] Bangladesh is among the most densely populated countries in the world, and shares land borders with India to the west, north, and east, and Myanmar to the southeast; to the south it has a coastline along the Bay of Bengal. It is narrowly separated from Bhutan and Nepal by the Siliguri Corridor; and from China by the Indian state of Sikkim in the north. Dhaka, the capital and largest city, is the nation's political, financial and cultural center. Chittagong, the second-largest city, is the busiest port on the Bay of Bengal. The official language is Bengali, one of the easternmost branches of the Indo-European language family. 
\section{History}
this is a regular text\\
\textbf{this is a regular text}\\
\textit{this is a regular text}\\
\textbf{\textit{this is a regular text}}\\

\section{Equation}
\subsection{Inline}

$x= 2 $
\[ x=2 \]

\[
\sigma
\Sigma
\]

\[
\omega
\Omega
<
\leq
\geq
T_{i_1,i_2,\dots,i_n}
\]

\begin{equation}
    x = \frac{\sqrt{\sin{\theta}}+{\cos{\theta}}+T_i}{e^{-20}}
\end{equation}

\begin{equation}
    x = \frac{-b\pm\sqrt{b^{2}-4ac}}{2a}
\end{equation}

\begin{equation}
    1 + 2 + 3 + \dots + n = 
    \frac{n(n+1}{2} = \sum_{i=1}^{i=N}i
\end{equation}
\section{Description}
\subsection{Dataset}
\subsection{Tables}
\subsection{List}
\subsubsection{Unordered list}
\begin{itemize}
    \item CSE
    \item EEE
\end{itemize}

\subsubsection{Ordered List}
\begin{enumerate}
    \item CSE 300
    \item CSE 308
    \begin{enumerate}
        \item[*] Latex Introduction
    \end{enumerate}
\end{enumerate}

\subsubsection{Ordered list using outline}

\subsubsection{Simple Table}
See the table \ref{tab:Multi col table}
\begin{table}[]
    \centering
    \begin{tabular}{|r|c|c|}
        \hline
        col1 & col2 & col3\\
        \hline
        1 & 2 & 3\\
        \cline{1-2}
        4 &  & 6\\
        \hline
    \end{tabular}
    \caption{Simple table}
    \label{tab:Multi col table}
\end{table}

\begin{table}[]
    \centering
    \begin{tabular}{|r|c|c|}
        \hline
        col1 & col2 & col3\\
        \hline
        \multicolumn{2}{|c|}{1} & 4\\
        \hline
        \multirow{2}{*}{4}  & 5 & 6\\
        \cline{2-3}
        
         & 8  & 9\\
        \hline
        \multicolumn{2}{|c|}{\multirow{2}{*}{10}} & 12\\
        \cline{3-3}
        \multicolumn{2}{|c|}{} & 17 \\
        \hline
        
    \end{tabular}
    \caption{Multi col table}
    \label{tab:Multi col table}
\end{table}

\section{Image}

See the image \ref{fig:my_label}
\begin{figure}
    \centering
    \begin{subfigure}{\textwidth}
     \includegraphics[width = \linewidth]{rickroll.jpg}
    \caption{Caption}
    \label{fig:my_label}
    \end{subfigure}
   \begin{subfigure}{\textwidth}
     \includegraphics[width = \linewidth]{rickroll.jpg}
    \caption{Caption}
    \label{fig:my_label}
   \end{subfigure}
\end{figure}

\section{Bibliography}
We used info from \cite{fuller1960behavior}
\printbibliography

\end{document}
